\section{Types of latency}

At this point our analysis requires defining the different ways data
transfers may be affected by latency issues and, if possible, the
signatures of these latency types.  We make no claim to completeness
and, at this stage, we just focus on three main cases.

\subsubsection*{Tails}
First of all let's consider block transfers that have ``tails''. That
is: one or few files that take much longer to get transferred than the
rest of the block. We can look for this type of latency by selecting
transfers in which the time needed for moving the last 5\% of bytes is
larger than a given threshold $\delta$.  To prevent very large blocks
from being included even if they have no real latency issues we may
add to $\delta$ an offset which depends from the size of the block and
a reference speed parameter $v$.  In formulas

\begin{equation}
{\Delta}T_{last 5\%} > \delta + \frac{S}{20 v}
\label{eq:early-stuck}
\end{equation}
where ${\Delta}T_{last 5\%}$ is the time of the last 5\% of replicas and
$S$ the size of the block. Sensible values for the parameters are
$v=5MB/s$ and $\delta=10h$.

\subsubsection*{Early Stuck}
Then we can consider block transfers that begin with serious
performance issues and start flowing properly only after some time,
presumably once such issues have been fixed. We may call such
transfers ``early stuck''.  To find this type of latency we can select
transfers in which the time for the first replica is larger than a
given threshold $\delta$.  An offset which depends from the average
size of the files and the reference rate parameter $v$ will prevent
blocks with very big files to show up even if they have no latency
issues. In formulas

\begin{equation}
{\Delta}T_{1st} > \delta + \frac{\langle S \rangle}{v}
\label{eq:early-stuck}
\end{equation}
where ${\Delta}T_{1st}$ is the time of the first replica and $\langle
S \rangle$ the average file size in the datablock. We can use again
the values above as good estimations of $v$ and $\delta$.

\subsubsection*{Many Small Blocks}
If we take large datasets with many small (few files) blocks, we can
consider the latency caused, at the dataset level, by some stuck
blocks. This is much like the latency tails discussed above but
replacing files by blocks and blocks by datasets.  This is another
important latency type that we will not be able to analyze since it
affects blocks of a type that we ruled out of our sample (they are
below the size limit).

\bigskip
Despite this list of data types is clearly incomplete, we will see
that it covers most of the real-life cases. In what follows we will in
fact proceed in a more detailed analysis of the transfers affected by
the latency types defined in this section. We will see which are their
causes and their consequences on the CMS operations.
