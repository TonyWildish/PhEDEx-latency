\section{Latency of datasets with many small blocks}
\label{sec:smallblocks}

In all CMS processing activities, datasets - to be intended as
collections of files with specific physics content - are produced via
the contribution of multiple WLCG sites supporting the CMS
workflows. In most cases, the contributing sites are not the final
destination where the dataset is supposed to be stored. Hence, a
PhEDEx subscription is made to ship all data to a tape endpoint - and
possibly to some disk endpoints - chosen according to CMS
policies. However, some of the blocks might be located at sites having
temporary network or storage problems which can cause a latency in
the overall data placement task.

Big datasets composed of many small blocks (few files per block) are
particularly exposed to this kind of issues as the sources of blocks
are multiple and may be particularly heterogeneous. This is much
similar to the latency issues described in sec.~\ref{sec:tails} but on
dataset/fileblock level rather than on fileblock/file level.

As we said in sec.~\ref{sec:types}, the way our final set of latency
data was selected does not allow us to study this kind of
latencies. However, one way to detect these kind of latencies among
the raw latency data provided by PhEDEx may be to select small group,
aggregate information by dataset, calculate the transfer rate
distribution of bocks within a dataset and look at the slow tails of
such distributions. We leave this analysis for future work.

