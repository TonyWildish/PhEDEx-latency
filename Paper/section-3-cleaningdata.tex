\section{Cleaning data and defining variables}
\label{sec:cleaning}

In the last two years, thanks to the instrumentation detailed in the
previous session, the PhEDEX system collected records about the
latency of roughly 3 million block subscriptions.  This data include
several useful informations:

\begin{itemize}
\item block informations: block, number of files, size, timestamps of
    the block opening and closing;
\item subscription informations: destination site, custodiality flag,
    total time of suspension, creation date of the subscription,
    priority;
\item source sites informations: site from which most files where
    transferred and number of files transferred from this site;

\item informations about the transfer execution: timestamps of the first
    request, the first transfer, the 25\%/50\%/75\%/95\% completion and of
    the last transfer. The total number of transfers attempts.
\end{itemize}

This huge amount of records represents the raw input of our
analysis. Before starting the actual analytic work, however, this set
of data has to undergo some cleaning and processing steps in order to
remove the items of no interest and define useful derived observables.

First of all we cleaned ill defined data, that is records with missing
or inconsistent entries. These are mostly issued from test transfers
and represent roughly the 5\% of the total amount.

In the remaining set there are roughly 780,000 transfers that took
place while the block was still open and growing in size. These would
actually be well defined targets for our analysis but their treatment
may render the whole process uselessly complex. Therefore, for the
time being, we decided to remove these entries from the set. For the
same reason we decided to remove all the 62,000 transfers entries
which have been suspended during their execution.

We also defined a cutoff of 3 hours on the transfer time. The idea
behind this choice is that, seen the typical time scale of data
transfers in CMS, if a transfer takes less than 3 hours from the
subscription to the completion we can, arbitrarily but sensibly, argue
that it is not a candidate for having latency problems. This cutoff
removes roughly 960,000 items.

At this point we are left with 1 million transfers records among which
we have to point out those having latency issues.  In order to
determine the signal of a latency problem we defined few ``skew"
variables showing the transfer rate ratio between the time spent in a
small portion (last 5\% or first 25\%) and the X percent from the
beginning or to the end of a transfer.  More precisely, we define the
following 4 sets of variables

\begin{itemize}
\item skew $X$ variables: 
  \begin{equation}
  Skew_X=\frac{\left( \mbox{time spent transferring the LAST 5 percent of the files} \right)}{ \left( \mbox{time spent transferring the FIRST } X \mbox{ percent of the files} \right)} * \frac{X}{5}
  \label{eq:skew}
   \end{equation}
   where $X=25,50,75,95$;
\item skew last $X$ variables: 
  \begin{equation}
   SkewLast_X=\frac{\left( \mbox{time spent transferring the LAST 5 percent of the files}\right)}{\left(\mbox{time spent transferring the LAST } X \mbox{ percent of the files}\right)} * \frac{X}{5}
  \label{eq:skewlast}
  \end{equation} 
   where $X=25,50,75$;
\item reverse skew  $X$ variables: 
  \begin{equation}
    RSkew_X=\frac{\left( \mbox{time spent transferring the FIRST 25 percent of the files}\right)}{\left( \mbox{time spent transferring the FIRST }X \mbox{ percent of the files}\right)} *  \frac{X}{25} 
  \label{eq:revskew}
  \end{equation}  
   where $X=50,75,95$;
\item reverse skew last $X$ variables: 
  \begin{equation}
    RSkewLast_X=\frac{\left(\mbox{time spent transferring the FIRST 25 percent of the files}\right)}{ \left(\mbox{time spent transferring the LAST }X\mbox{ percent of the files}\right)} * \frac{X}{25} 
  \label{eq:revskewlast}
  \end{equation}
   where $X=5,25,50,75$.
\end{itemize}

On a transfer which is ideally running at a constant rate all these
variables have value one. Thus, one or more of the ``skew" variables
that significantly differs from unity can be considered a good signal
of a transfer having latency issues. In particular this is the signal
of a latency ``tail" in a generalized sense, that is a section of the
transfer which has been significantly slower than the others.

This led us to a further data cleaning. Tails analysis requires in
fact defined values for the ``skew" variables as well as data blocks
which are big enough to have a ``bulk" and a ``tail". Thus, data for
this analysis was further skimmed keeping only blocks with more than 5
files, size larger than 300GB and with defined values for all the
``skew" variables. This leaves us with a final set of roughly 42,000
transfers entries which are the actual basis of our latency tails
analysis.

