\section{Instrumenting PhEDEx to collect latency data}
\label{sec:instrumenting}

The atomic unit for transfer operations in PhEDEx is the file block:
an arbitrary group of O(100-1000) files in the same dataset.  To
achieve scalability, PhEDEx doesn't keep a permanent record of the
states of individual files: all information is aggregated at the level
of block after transfers are completed.

This level of detail is sufficient for replica location, but it is not
enough to identify problems that increase latency in block transfers:
for example, there is no way to distinguish between the case of a
high-latency transfer proceeding at a low regular rate, and the case
of a transfer for which the latency is dominated by a few stuck files
in the ``transfer tail''.

For this reason, in 2012 we instrumented the central agents of PhEDEx
with a detailed latency monitoring system \cite{phedexlatency},
collecting file-level information on transfers in historical
monitoring tables to complement block-level information, and providing
corresponding Data Service APIs to retrieve the monitoring data for
further analysis.

The BlockAllocator agent that is responsible for monitoring
subscriptions records the timestamps of the main events related to
block completion in the t\textunderscore dps\textunderscore
block\textunderscore latency table: the time when the block was
subscribed and the time when the last file in the block was
successfully replicated at destination.  The difference between these
two timestamps can be defined as the total latency for block
replication as experienced by users, which may also include time spent
while the subscription was manually suspended by an operator.  In this
case, the agent also measures and records in the table the time
elapsed during the suspension, and subtracting this value from the
total latency we measure what we define as the ``PhEDEx latency'',
i.e. introduced by PhEDEx itself and the underlying transfer
infrastructure rather than human intervention.


The FilePump agent responsible for collecting the results of transfer
tasks records a summary of the transfer history of each file in the
block in the t\textunderscore xfer\textunderscore file\textunderscore
latency table, including the time when the file was first activated
for routing, the time of the first transfer attempt and of the final
successful transfer attempt, the number of transfer attempts needed
for the file to arrive at destination, as well as the source node of
the first and last transfer attempts (which may be different if the
transfer was rerouted).



For performance reasons, the entries in these live tables are archived
to historical tables after the transfer is completed: file-level
statistics into t\textunderscore log\textunderscore
file\textunderscore latency, which is eventually cleaned up after 30
days, and block-level statistics into t\textunderscore
log\textunderscore block\textunderscore latency, in which they are
kept indefinitely and integrated with additional events related to
block completion:
\begin{itemize}
\item the time when the first file in the block was routed for transfer
\item the time when the first file in the block was successfully
  replicated at destination
\item the times when 25\%/50\%/75\%/95\% of the files in the block
  were replicated at destination
\item the time when the last file in the block was successfully
  replicated at destination
\end{itemize}
In addition, we record the total number of transfer attempts needed to
transfer all files in the block, as well as the source node for the
majority of files in the block.
